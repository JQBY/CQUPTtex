%% chapter-1: 绪论部分
%% 介绍本研究课题的学术背景及理论与实际意义,国内外文献综述,
%% 本研究课题的来源及主要研究内容。

\chapter{论文结构及文字格式}
学士学位论文应能表明作者确已较好地掌握了本门学科的基础理论、专门知识和基本技能,并从事科学研究工作或独立担负专门技术工作的初步能力。论文的文字表述应实事求是、客观真切、合乎逻辑、层次分明、简练可读。凡引用他人观点、方案、资料、数据、图表等,无论是纸质或电子版,均应详加注释。论文结构和文字格式应规范。\textcolor{red}{(注:参考此格式进行排版。标题名称,自己根据情况进行修改。)}
\section{论文结构}
论文由前置部分、主体部分和附录部分构成。前置部分包括封面、封二、中英文摘要、中英文关键词、目录、图录、表录、注释表、附件清单等;主体部分包括引文、正文、结论、致谢、参考文献等;附录部分为论文主体部分的补充,用于编排论文相关的资料,如设计说明、软件源代码、设计图纸、测试报告、英文翻译等。

论文应根据内容的相对独立性划分各章,每章的内容精简后可作为期刊论文发表,各章的顺序安排应考虑论文内容的逻辑性。各章之间应重新分页,章的标题在起始页。

论文正文是论文的核心部分,占主要篇幅。由于研究工作涉及的学科、选题、研究方法、工作进程、结果表达方式等有很大差异,对正文内容不作统一规定,但正文应对研究内容及成果进行较全面、客观的理论阐述,应着重指出研究内容中的创新、改进与实际应用之处。


\section{学位论文中的引言}
\subsection{引言的目的}
国家标准GB7713-87规定:引言(或绪论)简要说明研究工作的目的、范围、相关领域的前人工作和知识空白、理论基础和分析、研究设想、研究方法和实验设计、预期结果和意义等。应言简意赅,不要与摘要雷同,不要成为摘要的注释。一般教科书中有的知识,在引言中不要赘述。

学位论文需要反映出作者确已掌握了坚实宽广的基础理论和系统深入的专门知识,具有开阔的科学视野,对研究方案作了充分论证,因此,有关历史回顾和前人工作的文献综述,以及理论分析都可以放在引言里。

引言的目的是给出作者进行本项工作的原因,希望达到的目的。因此应给出必要的背景材料,让对这一领域并不特别熟悉的读者能够了解进行这方面研究的意义,前人已达到的水平,已解决和尚待解决的问题,引出要研究的内容,介绍通过研究取得的成果和主要创新之处。



\subsection{引言构成及写作要求}
引言的构成及写作要求如表~\ref{tab:2.1}所示。\textcolor{red}{(点表2.1可以跳转到该表)}


\begin{table}[h] %htbp 可以调整图或表的位置
    \small       %设置字号五号 
	\centering
	\caption[表2.1]{引言构成及写作要求}
	\label{tab:2.1}
    	\begin{tabular}{m{7cm}|m{7cm}}
    	\hline 
        \makecell[c]{\textbf{基本项目}} 	&\makecell[c]{\textbf{主要内容}}\\ 
        \hline 
        研究的必要性(存在的问题)	& 原来存在的问题,提出了什么要求,说明这项研究的意义 \\ 
        \hline 
        历史的回顾	&  对于存在的问题,前人进行过怎样的研究,介绍其大概情形\\ 
        \hline 
        前人研究中存在的欠缺	& 考察了前人的研究之后,发现了什么欠缺,还可以介绍自己研究的动机 \\ 
        \hline 
        写作论文的目的和作者的想法	& 写作目的和涉及的范围,研究结果的适用范围,研究者有什么建议,研究的新特点是什么 \\ 
        \hline 
        处理方法和研究结果简介(具体数据)	& 引用从具体数值计算出的数据,介绍研究的经过和结果 \\ 
        \hline 
        \end{tabular} 
\end{table}




\section{本章小结}
本科毕业论文由前置部分、主体部分和附录部分构成,撰写论文时需按此模板要求和格式编排。

