\chapter{参考文献的标注和要求}

学位论文必须要列参考文献,以说明著述内容的科学根据和出处,进而方便读者扩展性阅读的查找。若引用他人成果,应列出引文出处,以尊重他人的科学研究成果[2,15]。\textcolor{red}{(注:仅参考此格式进行排版。参考文献具体格式及排版见后面“参考文献”部分。)}

\section{参考文献的重要性}

参考文献反映论文作者的科学态度和论文具有真实、广泛的科学依据,也反映论文的起点和深度。方便论文作者与前人的成果区别开来,是对他人劳动成果的尊重。方便读者检索和查找有关资料。有利于节省论文篇幅,有助于科技情报人员进行情报研究和计量学研究。

\section{顺序编码体系}

参考文献标著录有如下4种体系:著者-出版年体系(Harvard体系),顺序编码体系,数字字母混合体系和出版年顺序体系[4]。我校本科学位论文的参考文献标注采用顺序编码体系。即参考文献以文献在整个学位论文中出现的次序用[1]、[2]、[3]…形式统一排序、依次列出,置于文中提及的文字末尾的右上角,视引文表标注情况,置于标点符号前或后[4]。

\subsection{正文中引用文献的标注方法}

正文中引用文献的标示应置于所引内容最后一个字的右上角,所引文献编号用阿拉伯数字置于方括号“[ ]”中,用小4号字体的上角标,引用单篇文献时如例1所示;引用两篇文献时,各篇文献序号置于同一个方括号内,其间用逗号(不是顿号)分开,如例2所示,如果连续序号多于两个以上时,可用范围号“~”(中文还可用全身“—”,外文用en-dash“-”)连接起讫序号,如例3所示;如果文献序号作为叙述文字的一部分,则文献号与正文平排,并且每条文献都要加方括号,如例4和例5所示;如果同一文献在文中不同处被重复引用,全文只在其第一次出现时标应标的序号,以后各处均标这同一序号;若必须标出引文页码,可把页码标在方括号外,如例6所示,也可用其他明确的方式标出。

例1:……,表明已低到2500 m的高度[2],……。

例2:原位生成的TiB主要有针状或晶须状[21,22]

例3:复杂网络是当今学术界的研究热点[3-6],早期的研究结果[2,4,6-9] 表明,……。

例4:文献[2]指出,此高度已低到2500 m。

例5:紫色土壤主要分布在我国西南地区(参见文献[11]、[12]、[32])。

例6:……,表明已低到2500 m的高度[2,35],……。

不得将引用文献标示置于各级标题处。


\subsection{文后参考文献表的著录方法}


按论文中引用的顺序号排列参考文献,不按著者,不分语种。多著者时,著者间用西文逗号隔开,只列前3人,后加“等(et al.)”。在句中,凡是西文符号后应空半格。

参考文献的著录格式严格按照以下形式书写(含标点符号):

\textbf{1)专著:}作者.书名.版本(第1版不著录)[M].出版地:出版者,出版年:引用起止页码.

\textbf{2)译著:}作者.书名[M].译者,译.出版地:出版者,出版年: 引用起止页码.

\textbf{3)期刊:}作者.题名[J].刊名,出版年份,卷(期):起止页码.

\textbf{4)会议论文集:}作者.题名[C]// 编者.论文集名.出版地:出版者,出版年:起止页码.

\textbf{5)学位论文:}作者.题名:[D].保存地:保存者,年份.

\textbf{6)专利文献}:专利申请者.题名.专利国别,专利号[P].公告日期或公开日期.

\textbf{7)标准:}责任者.标准代号标准名称[S].出版地:出版者,出版年.

\textbf{8)电子文献标注格式:}主要责任者.题名: 其它题名信息[文献类型标志/文献载体标志].出版地: 出版者, 出版年(更新或修改日期)(引用日期).获取和访问路径.

参考文献著录时还应注意:参考文献中的外文作者名、外文刊名的缩写一律不用缩写点。外文著者一律用姓在名前,采用首字母缩写(中国人用全名不缩写)。姓和名之间不加逗号,名2个以上大写首字母,2名间空一格。文献作者3名以内全部列出,4名以上则列前3人,后加“et al”。各著者间不加“and”、“和”等,应用逗号分开。外文题名第一个单词首字母大写,其余单词(专有名词除外)均不大写。外文刊名应按国际标准规定缩写,不加缩写点。

特别提醒,对西文作者,正文中引用时应遵循西文规范,即名在前姓在后,缩写部分要加缩写点。以参考文献[10]为例,其参考文献条目为:

[10]\quad Atzori L, Iera A, Morabito G. The internet of things: a survey[J]. Computer Networks, 2010, 54(15): 2787 -2805.

正文中引用时的文字为:“L. Atzori等人对物联网的进展进行了综述,指出……”。西文人名乱用、不当使用是常见的写作问题。

通过查非发现,大部分论文参考文献格式都存在各种问题,应该严格规范执行。更多样例见本模板参考文献部分。如有未尽之处,可参看发表的重邮学报论文参考文献标注样例。需要特别说明的是,由于不是所有已发表论文或高年级撰写的学位论文都严格执行了同一规范,请勿将不规范的格式当做模仿对象,以免给自己带来后续问题。

\section{参考文献要求}

1)所有被引用文献均要列入参考文献中,必须按顺序标注,但同一篇文章只用一个序号。

2)参考文献数量不得少于15篇,各二级学院应按不同专业提出外文参考文献具体要求,教科书和硕士学位论文不得多于5本,其中外文文献不少于5条。参考文献中近三年的文献数一般应不少于总数的1/3,并应有近两年的参考文献。(网上参考文献和各类标准不包含在上述规定的文献数量之内)。

3)教材、产品说明书、未公开发表的研究报告(著名的内部报告如PB、AD报告及著名大公司的企业技术报告等除外)等通常不宜作为参考文献引用。一些未公开发表的内容引用时,可以采用注释如脚注的方式。

4)引用网上参考文献时,应注明该文献的准确网页地址。因为网上文献大都不规范,除特殊情况,原则上不引用,尽量用脚注方式给出。如测试数据来源网址,在正文首次出现时均可直接脚注或正文中给出出处。

5)序号应按文献在论文中的被引用顺序编排。换行时与作者名第一个字对齐。若同一文献中有多处被引用,则要写出相应引用页码,各起止页码间空一格,排列按引用顺序,不按页码顺序。

参考文献内容中文宋体,英文Times New Roman,小4号宋体,1.5倍行距。

Word也提供了参考文献引用的基本功能,但维护一致性比较困难。目前有较多的参考文献管理软件或工具,各有利弊,可根据情况使用,相互交流使用经验,更好地做好文献管理。

\section{本章小结}

介绍了参考文献的标注方法、著录方法和相关要求。










