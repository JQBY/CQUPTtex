\chapter{引言}
\textcolor{red}{绪论(也称引言)简要说明研究工作的目的、范围、相关领域国内外研究现状、研究目标、研究设想和内容、研究和实验方法、预期结果和意义,以及论文的章节安排等。力求言简意赅,不要与摘要雷同,也不要叙述教科书中的知识。(注:每章单独另起一页。参考此格式进行排版。文中红色字体为重点要求,黑色字体仅为格式参考。)}
\section{研究背景和意义}
学位论文是学生从事科研工作的成果的主要表现,它集中表明了作者在研究工作中获得的新的发明、理论或见解,是申请学位的重要依据,也是科研领域中的重要文献资料和社会的宝贵财富。学位论文应能表明作者已在本门学科上掌握了坚实的基础理论和系统的专业知识,并对所研究课题有新的见解,有从事科学研究工作或独立担负专门技术工作的能力 \upcite{学位论文编写规则}。\textcolor{red}{(注:参考文献按照引用顺序首次出现,且文中为右上标)}

本模板主要参照《学位论文编写规则》(GB/T7713.1-2006,中国国家标准局2006年发布并实施)\upcite{学位论文编写规则}、科学出版社出版的《作者编辑手册》\upcite{汪继祥2004科学出版社作者编辑手册}、全国科学道德和学风建设宣传教育领导小组制定的《科学道德与学风建设宣讲参考大纲(试用本)》(2011年11月)\upcite{全国科学道德和学风建设宣讲教育领导小组2012科学道德与学风建设宣讲参考大纲}、《文后参考文献著录规则》(GB/T7714-2005,中国国家标准局2005年发布并实施)\upcite{曹敏2005新版《文后参考文献著录规则》解析}等制定。

部分范例来自《障碍环境中Swarm突现计算模型研究及行为控制》\upcite{王兰芬2010Swarm} 等重庆邮电大学硕士学位论文。
\textcolor{red}{(注:根据自己的课题内容,可设置若干3级目录。3级目录下按照1)→(1)→\textcircled{1}...的层次编号。)}
\section{国内外研究现状}
\subsection{国外研究现状}
\textbf{1)分类号}

分类号指中图分类号,是指采用《中国图书馆分类法》(原称《中国图书馆图书分类法》,简称《中图法》)对科技文献进行主题分析,并依照文献内容的学科属性和特征,分门别类地组织文献,所获取的分类代号。采用1999年出版的第四版《中图法》可以在http://www.33tt.com/tools/ztf(中国图书馆分类法中图分类号查询系统)或http://lib.jzit.edu.cn/sjk/tsflf/index.htm(中图法第四版计算机辅助分类查询系统)中查询。填写要求:要求分类细分到22个大类代码后三位数字。如:TN929。

\textbf{2) UDC编号	}

UDC即国际十进分类法(Universal Decimal Classification),是国际通用的多文种综合性文献分类法。UDC采用单纯阿拉伯数字作为标记符号。它用个位数(0\textasciitilde9)标记一级类,十位数(00\textasciitilde99)标记二级类,百位数(000\textasciitilde999)标记三级类,以下每扩展(细分)一级,就加一位数。每三位数字后加一小数点。如电气工程类的论文,其UDC编号为:621.3。


\subsection{国内研究现状}
论文中文题名是以最恰当、最简明的词语,反映学位论文最重要的特定内容的逻辑组合。题名用词应有助于选关键词和编制题录、索引等二次文献,可以提供检索的特定实用信息。题名应恰当简洁,一般不超过25个字。题名应避免使用不常见的缩写词、首字缩写字、字符、代号及公式等。题名语意未尽时,可以用副标题补充说明论文中的特定内容[1]。题名中文宋体,英文Times New Roman小二号字。


\section{主要内容和工作安排}
写出论文的主要工作内容,并逐一介绍每章的内容安排。全文共分为5章,内容结构安排如下:

第1章为引言,引入课题的研究背景及意义….

第2章是天线基本理论分析,….

第3章是设计仿真,….

第4章为优化与分析,….

第5章作为论文的结束语,总结毕业设计工作,提出可以在今后继续深入研究的方向。


